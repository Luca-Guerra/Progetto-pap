\documentclass{article}
\usepackage{graphicx}
\usepackage[utf8]{inputenc}
\begin{document}

\title{Progetto Programmazione avanzata e paradigmi \\ Indicizzazione documenti}

\author{Luca Guerra}

\maketitle

\section{Introduzione}
Si vuole sviluppare l'indicizzazione di diversi documenti di testo, in modo da sapere velocemente quali documenti contengono certe parole.

\section{Visione del problema}
Si possono individuare i seguenti blocchi di lavoro (Task) per il problema in esame:
\begin{itemize}
  \item individuare i file da indicizzare
  \item indicizzare questi file
  \item cercare l'occorrenza cercata nei vari file
\end{itemize} 
I primi due Task possono essere realizzati in parallelo in maniera concorrente, mentre il terzo dovrà attendere il termine dei primi due (per evitare di dare una risposta incompleta).

\section{Soluzione al problema}

Per rendere il sistema più performante, ho deciso di sfruttare una \textbf{BlockingDeque}, questa verrà utilizzata come contenitore dei vari documenti da indicizzare. Il flusso di lavoro darà il seguente: \\
Allo start dell'indicizzazione avrò un processo che naviga tutto l'albero partendo dalla root ( basta un unico thread, perchè il lavoro di navigare la root sarà molto più veloce del lavoro di indicizzazione) aggiungerà file alla lista, i vari processi indicizzatori rimarranno in attesa di nuovi file, appena questi saranno presenti li prenderanno e inizieranno a indicizzare in una hashtable, in questo caso questa sarà la nostra memoria condivisa, e l'accesso a questa dovrà essere mutualmente esclusivo.
Per indicizzare nella hashtable avremo la parola come chiave, e come valore una lista di stringhe che conterranno tutti nomi dei documenti che contengono la stringa specificata.\\
Terminata questa prima parte, il sistema è pronto a rispondere alle varie query e consultando la tabella rispondere velocemente(anche questo verrà fatto da un solo thread).

\section{Conclusion}
Write your conclusion here.

\end{document}